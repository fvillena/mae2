\documentclass{article}

\usepackage[utf8]{inputenc}
\usepackage[spanish]{babel}
\usepackage[hidelinks]{hyperref}
\usepackage{amsfonts} 
\usepackage{amsmath}
\usepackage{listings}
\usepackage{graphicx}
\renewcommand{\lstlistingname}{Texto}
\lstset{
    inputencoding = utf8,  % Input encoding
    extendedchars = true,  % Extended ASCII
    literate      =        % Support additional characters
      {á}{{\'a}}1  {é}{{\'e}}1  {í}{{\'i}}1 {ó}{{\'o}}1  {ú}{{\'u}}1
      {Á}{{\'A}}1  {É}{{\'E}}1  {Í}{{\'I}}1 {Ó}{{\'O}}1  {Ú}{{\'U}}1
      {à}{{\`a}}1  {è}{{\`e}}1  {ì}{{\`i}}1 {ò}{{\`o}}1  {ù}{{\`u}}1
      {À}{{\`A}}1  {È}{{\`E}}1  {Ì}{{\`I}}1 {Ò}{{\`O}}1  {Ù}{{\`U}}1
      {ä}{{\"a}}1  {ë}{{\"e}}1  {ï}{{\"i}}1 {ö}{{\"o}}1  {ü}{{\"u}}1
      {Ä}{{\"A}}1  {Ë}{{\"E}}1  {Ï}{{\"I}}1 {Ö}{{\"O}}1  {Ü}{{\"U}}1
      {â}{{\^a}}1  {ê}{{\^e}}1  {î}{{\^i}}1 {ô}{{\^o}}1  {û}{{\^u}}1
      {Â}{{\^A}}1  {Ê}{{\^E}}1  {Î}{{\^I}}1 {Ô}{{\^O}}1  {Û}{{\^U}}1
      {œ}{{\oe}}1  {Œ}{{\OE}}1  {æ}{{\ae}}1 {Æ}{{\AE}}1  {ß}{{\ss}}1
      {ẞ}{{\SS}}1  {ç}{{\c{c}}}1 {Ç}{{\c{C}}}1 {ø}{{\o}}1  {Ø}{{\O}}1
      {å}{{\aa}}1  {Å}{{\AA}}1  {ã}{{\~a}}1  {õ}{{\~o}}1 {Ã}{{\~A}}1
      {Õ}{{\~O}}1  {ñ}{{\~n}}1  {Ñ}{{\~N}}1  {¿}{{?`}}1  {¡}{{!`}}1
      {°}{{\textdegree}}1 {º}{{\textordmasculine}}1 {ª}{{\textordfeminine}}1
      {£}{{\pounds}}1  {©}{{\copyright}}1  {®}{{\textregistered}}1
      {«}{{\guillemotleft}}1  {»}{{\guillemotright}}1  {Ð}{{\DH}}1  {ð}{{\dh}}1
      {Ý}{{\'Y}}1    {ý}{{\'y}}1    {Þ}{{\TH}}1    {þ}{{\th}}1    {Ă}{{\u{A}}}1
      {ă}{{\u{a}}}1  {Ą}{{\k{A}}}1  {ą}{{\k{a}}}1  {Ć}{{\'C}}1    {ć}{{\'c}}1
      {Č}{{\v{C}}}1  {č}{{\v{c}}}1  {Ď}{{\v{D}}}1  {ď}{{\v{d}}}1  {Đ}{{\DJ}}1
      {đ}{{\dj}}1    {Ė}{{\.{E}}}1  {ė}{{\.{e}}}1  {Ę}{{\k{E}}}1  {ę}{{\k{e}}}1
      {Ě}{{\v{E}}}1  {ě}{{\v{e}}}1  {Ğ}{{\u{G}}}1  {ğ}{{\u{g}}}1  {Ĩ}{{\~I}}1
      {ĩ}{{\~\i}}1   {Į}{{\k{I}}}1  {į}{{\k{i}}}1  {İ}{{\.{I}}}1  {ı}{{\i}}1
      {Ĺ}{{\'L}}1    {ĺ}{{\'l}}1    {Ľ}{{\v{L}}}1  {ľ}{{\v{l}}}1  {Ł}{{\L{}}}1
      {ł}{{\l{}}}1   {Ń}{{\'N}}1    {ń}{{\'n}}1    {Ň}{{\v{N}}}1  {ň}{{\v{n}}}1
      {Ő}{{\H{O}}}1  {ő}{{\H{o}}}1  {Ŕ}{{\'{R}}}1  {ŕ}{{\'{r}}}1  {Ř}{{\v{R}}}1
      {ř}{{\v{r}}}1  {Ś}{{\'S}}1    {ś}{{\'s}}1    {Ş}{{\c{S}}}1  {ş}{{\c{s}}}1
      {Š}{{\v{S}}}1  {š}{{\v{s}}}1  {Ť}{{\v{T}}}1  {ť}{{\v{t}}}1  {Ũ}{{\~U}}1
      {ũ}{{\~u}}1    {Ū}{{\={U}}}1  {ū}{{\={u}}}1  {Ů}{{\r{U}}}1  {ů}{{\r{u}}}1
      {Ű}{{\H{U}}}1  {ű}{{\H{u}}}1  {Ų}{{\k{U}}}1  {ų}{{\k{u}}}1  {Ź}{{\'Z}}1
      {ź}{{\'z}}1    {Ż}{{\.Z}}1    {ż}{{\.z}}1    {Ž}{{\v{Z}}}1
      % ¿ and ¡ are not correctly displayed if inconsolata font is used
      % together with the lstlisting environment. Consider typing code in
      % external files and using \lstinputlisting to display them instead.      
  }

\title{Control}
\author{Fabián Villena y Felipe Arias}
\date{Junio 2023}

\begin{document}

\maketitle

\section*{P1 (2 puntos): Preprocesamiento}

Usted tiene una lista de textos \texttt{T} extraídos desde sospechas diagnósticas de pacientes y se le pide que los preprocese\footnote{No se preocupe si el corpus preprocesado no queda exactamente igual al de la P2.} para poder representarlos en un paso siguiente mediante \textit{Bag-of-words}.

\begin{verbatim}
T = [
    ["Caries-dentinaria SUPERFICIAL y caries radicular"],
    ["Caries dentinaria profunda"],
    ["CARIES DEL CEMENTO, CARIES DEL ESMALTE"],
    ["quiste radicular??????"],
    ["cáncer del colon"]
]
\end{verbatim}

\subsection*{P1.1 (1,5 puntos)}

Desarrolle un algoritmo en pseudocódigo o \textit{Python} para preprocesar cada uno de los documentos del corpus \texttt{T}. Su preprocesamiento debe tomar en cuenta la eliminación de \textit{stop words} desde una lista elaborada por usted.

\subsection*{P1.2 (0,5 puntos)}

Desarrolle un algoritmo en pseudocódigo o \textit{Python} para \textit{tokenizar} cada uno de los documentos del corpus \texttt{T}.

\newpage

\section*{P2 (4 puntos): Semántica vectorial}

Se le presenta un corpus \texttt{D} de 5 documentos ya \textit{tokenizados} y se le pide represental los documentos mediante un \textit{vectorizador Booleano} y compararlos a través del \textit{índice de Jaccard}.

\begin{verbatim}
D = [
    ["caries", "dentinaria", "superficial", "y", "caries", "radicular"],
    ["caries", "dentinaria", "profunda"],
    ["caries", "del", "cemento", "caries", "del", "esmalte"],
    ["quiste", "radicular"],
    ["cáncer", "del", "colon"]
]
\end{verbatim}

El \textit{vectorizador Booleano} es una variante del método de representación \textit{Bag-of-Words} en donde si una palabra del vocabulario está presente en el documento se le asigna un 1 a esa dimensión y de lo contrario se le asigna un 0. En el caso de \textit{Bag-of-Words} cada dimensión estaría determinada por la frecuencia de la palabra en el documento.

Para poder calcular similaridades entre los vectores representados por el \textit{vectorizador Booleano} se puede utilizar el \textit{índice de Jaccard} $\mathcal{J}$, el cual se define como la intersección de los conjuntos dividido por la union de los conjuntos. La intersección de los conjuntos es la suma de los elementos que se encuentran en ambos conjuntos y la unión de los conjuntos es la suma de la cantidad de elementos que tiene cada conjunto.

Por ejemplo, para

\begin{align*}
	\vec{v_1} = [1,0,1,0,1] 
	\textrm{ y }            
	\vec{v_2} = [1,1,1,0,0] 
	\textrm{, }             
\end{align*}

\begin{align*}
	\mathcal{J}(\vec{v_1},\vec{v_2}) = \frac{\textrm{cantidad de elementos que están en } \vec{v_1} \textrm{ y en } \vec{v_2}}{\textrm{cantidad de elementos que están en } \vec{v_1} \textrm{ y/o en } \vec{v_2}} = \frac{2}{4} = 0.5 
\end{align*}

\newpage

\subsection*{P2.1 (2 puntos)}

Describa el algoritmo para calcular las representaciones de \texttt{D} mediante el \textit{vectorizador Booleano} en pseudocódigo o en \textit{Python} y complete el cuadro \ref{table:p11} con las representaciones de los documentos.

\begin{table}[h!]
	\centering
	\begin{tabular}{lllllllllllll}
		                                     & \rotatebox{90}{colon} & \rotatebox{90}{superficial} & \rotatebox{90}{del}   & \rotatebox{90}{radicular} & \rotatebox{90}{quiste} & \rotatebox{90}{y}     & \rotatebox{90}{profunda} & \rotatebox{90}{caries} & \rotatebox{90}{esmalte} & \rotatebox{90}{cemento} & \rotatebox{90}{dentina} & \rotatebox{90}{cáncer} \\ \cline{2-13} 
		\multicolumn{1}{l|}{\texttt{D}$[0]$} & \multicolumn{1}{l|}{} & \multicolumn{1}{l|}{}       & \multicolumn{1}{l|}{} & \multicolumn{1}{l|}{}     & \multicolumn{1}{l|}{}  & \multicolumn{1}{l|}{} & \multicolumn{1}{l|}{}    & \multicolumn{1}{l|}{}  & \multicolumn{1}{l|}{}   & \multicolumn{1}{l|}{}   & \multicolumn{1}{l|}{}   & \multicolumn{1}{l|}{}   \\[2ex] \cline{2-13} 
		\multicolumn{1}{l|}{\texttt{D}$[1]$} & \multicolumn{1}{l|}{} & \multicolumn{1}{l|}{}       & \multicolumn{1}{l|}{} & \multicolumn{1}{l|}{}     & \multicolumn{1}{l|}{}  & \multicolumn{1}{l|}{} & \multicolumn{1}{l|}{}    & \multicolumn{1}{l|}{}  & \multicolumn{1}{l|}{}   & \multicolumn{1}{l|}{}   & \multicolumn{1}{l|}{}   & \multicolumn{1}{l|}{}   \\[2ex] \cline{2-13} 
		\multicolumn{1}{l|}{\texttt{D}$[2]$} & \multicolumn{1}{l|}{} & \multicolumn{1}{l|}{}       & \multicolumn{1}{l|}{} & \multicolumn{1}{l|}{}     & \multicolumn{1}{l|}{}  & \multicolumn{1}{l|}{} & \multicolumn{1}{l|}{}    & \multicolumn{1}{l|}{}  & \multicolumn{1}{l|}{}   & \multicolumn{1}{l|}{}   & \multicolumn{1}{l|}{}   & \multicolumn{1}{l|}{}   \\[2ex] \cline{2-13} 
		\multicolumn{1}{l|}{\texttt{D}$[3]$} & \multicolumn{1}{l|}{} & \multicolumn{1}{l|}{}       & \multicolumn{1}{l|}{} & \multicolumn{1}{l|}{}     & \multicolumn{1}{l|}{}  & \multicolumn{1}{l|}{} & \multicolumn{1}{l|}{}    & \multicolumn{1}{l|}{}  & \multicolumn{1}{l|}{}   & \multicolumn{1}{l|}{}   & \multicolumn{1}{l|}{}   & \multicolumn{1}{l|}{}   \\[2ex] \cline{2-13} 
		\multicolumn{1}{l|}{\texttt{D}$[4]$} & \multicolumn{1}{l|}{} & \multicolumn{1}{l|}{}       & \multicolumn{1}{l|}{} & \multicolumn{1}{l|}{}     & \multicolumn{1}{l|}{}  & \multicolumn{1}{l|}{} & \multicolumn{1}{l|}{}    & \multicolumn{1}{l|}{}  & \multicolumn{1}{l|}{}   & \multicolumn{1}{l|}{}   & \multicolumn{1}{l|}{}   & \multicolumn{1}{l|}{}   \\[2ex] \cline{2-13} 
	\end{tabular}
	\caption{Representaciones vectoriales de \texttt{D}.}
	\label{table:p11}
\end{table}

\subsection*{P2.2 (1,5 puntos)}

Describa el algoritmo n pseudocódigo o en \textit{Python} para calcular las similaridades entre cada uno de los documentos de \texttt{D} utilizando el \textit{índice de Jaccard} y complete el cuadro \ref{table:p12} con las similaridades.

\begin{table}[h!]
	\centering
	\begin{tabular}{llllll}
		\texttt{D}$[0]$         & \texttt{D}$[1]$        & \texttt{D}$[2]$        & \texttt{D}$[3]$        & \texttt{D}$[4]$        &                 \\ \cline{1-5}
		\multicolumn{1}{|l|}{1} & \multicolumn{1}{l|}{}  & \multicolumn{1}{l|}{}  & \multicolumn{1}{l|}{}  & \multicolumn{1}{l|}{}  & \texttt{D}$[0]$ \\[2ex] \cline{1-5}
		\multicolumn{1}{l|}{}   & \multicolumn{1}{l|}{1} & \multicolumn{1}{l|}{}  & \multicolumn{1}{l|}{}  & \multicolumn{1}{l|}{}  & \texttt{D}$[1]$ \\[2ex] \cline{2-5}
		                        & \multicolumn{1}{l|}{}  & \multicolumn{1}{l|}{1} & \multicolumn{1}{l|}{}  & \multicolumn{1}{l|}{}  & \texttt{D}$[2]$ \\[2ex] \cline{3-5}
		                        &                        & \multicolumn{1}{l|}{}  & \multicolumn{1}{l|}{1} & \multicolumn{1}{l|}{}  & \texttt{D}$[3]$ \\[2ex] \cline{4-5}
		                        &                        &                        & \multicolumn{1}{l|}{}  & \multicolumn{1}{l|}{1} & \texttt{D}$[4]$ \\[2ex] \cline{5-5}
	\end{tabular}
	\caption{Similaridades entre los documentos del corpus \texttt{D}.}
	\label{table:p12}
\end{table}

\subsection*{P2.3 (0,5 puntos)}

Explique las similaridades entre los documentos \texttt{D}[0] - \texttt{D}[1] y \texttt{D}[0] - \texttt{D}[4] tomando en cuenta el contenido de los documentos y los valores de $\mathcal{J}$.

\end{document}
