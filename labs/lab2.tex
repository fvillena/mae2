\documentclass{article}

\usepackage[utf8]{inputenc}
\usepackage[spanish]{babel}
\usepackage{hyperref}
\usepackage{listings}
\lstset{
    inputencoding = utf8,  % Input encoding
    extendedchars = true,  % Extended ASCII
    literate      =        % Support additional characters
      {á}{{\'a}}1  {é}{{\'e}}1  {í}{{\'i}}1 {ó}{{\'o}}1  {ú}{{\'u}}1
      {Á}{{\'A}}1  {É}{{\'E}}1  {Í}{{\'I}}1 {Ó}{{\'O}}1  {Ú}{{\'U}}1
      {à}{{\`a}}1  {è}{{\`e}}1  {ì}{{\`i}}1 {ò}{{\`o}}1  {ù}{{\`u}}1
      {À}{{\`A}}1  {È}{{\`E}}1  {Ì}{{\`I}}1 {Ò}{{\`O}}1  {Ù}{{\`U}}1
      {ä}{{\"a}}1  {ë}{{\"e}}1  {ï}{{\"i}}1 {ö}{{\"o}}1  {ü}{{\"u}}1
      {Ä}{{\"A}}1  {Ë}{{\"E}}1  {Ï}{{\"I}}1 {Ö}{{\"O}}1  {Ü}{{\"U}}1
      {â}{{\^a}}1  {ê}{{\^e}}1  {î}{{\^i}}1 {ô}{{\^o}}1  {û}{{\^u}}1
      {Â}{{\^A}}1  {Ê}{{\^E}}1  {Î}{{\^I}}1 {Ô}{{\^O}}1  {Û}{{\^U}}1
      {œ}{{\oe}}1  {Œ}{{\OE}}1  {æ}{{\ae}}1 {Æ}{{\AE}}1  {ß}{{\ss}}1
      {ẞ}{{\SS}}1  {ç}{{\c{c}}}1 {Ç}{{\c{C}}}1 {ø}{{\o}}1  {Ø}{{\O}}1
      {å}{{\aa}}1  {Å}{{\AA}}1  {ã}{{\~a}}1  {õ}{{\~o}}1 {Ã}{{\~A}}1
      {Õ}{{\~O}}1  {ñ}{{\~n}}1  {Ñ}{{\~N}}1  {¿}{{?`}}1  {¡}{{!`}}1
      {°}{{\textdegree}}1 {º}{{\textordmasculine}}1 {ª}{{\textordfeminine}}1
      {£}{{\pounds}}1  {©}{{\copyright}}1  {®}{{\textregistered}}1
      {«}{{\guillemotleft}}1  {»}{{\guillemotright}}1  {Ð}{{\DH}}1  {ð}{{\dh}}1
      {Ý}{{\'Y}}1    {ý}{{\'y}}1    {Þ}{{\TH}}1    {þ}{{\th}}1    {Ă}{{\u{A}}}1
      {ă}{{\u{a}}}1  {Ą}{{\k{A}}}1  {ą}{{\k{a}}}1  {Ć}{{\'C}}1    {ć}{{\'c}}1
      {Č}{{\v{C}}}1  {č}{{\v{c}}}1  {Ď}{{\v{D}}}1  {ď}{{\v{d}}}1  {Đ}{{\DJ}}1
      {đ}{{\dj}}1    {Ė}{{\.{E}}}1  {ė}{{\.{e}}}1  {Ę}{{\k{E}}}1  {ę}{{\k{e}}}1
      {Ě}{{\v{E}}}1  {ě}{{\v{e}}}1  {Ğ}{{\u{G}}}1  {ğ}{{\u{g}}}1  {Ĩ}{{\~I}}1
      {ĩ}{{\~\i}}1   {Į}{{\k{I}}}1  {į}{{\k{i}}}1  {İ}{{\.{I}}}1  {ı}{{\i}}1
      {Ĺ}{{\'L}}1    {ĺ}{{\'l}}1    {Ľ}{{\v{L}}}1  {ľ}{{\v{l}}}1  {Ł}{{\L{}}}1
      {ł}{{\l{}}}1   {Ń}{{\'N}}1    {ń}{{\'n}}1    {Ň}{{\v{N}}}1  {ň}{{\v{n}}}1
      {Ő}{{\H{O}}}1  {ő}{{\H{o}}}1  {Ŕ}{{\'{R}}}1  {ŕ}{{\'{r}}}1  {Ř}{{\v{R}}}1
      {ř}{{\v{r}}}1  {Ś}{{\'S}}1    {ś}{{\'s}}1    {Ş}{{\c{S}}}1  {ş}{{\c{s}}}1
      {Š}{{\v{S}}}1  {š}{{\v{s}}}1  {Ť}{{\v{T}}}1  {ť}{{\v{t}}}1  {Ũ}{{\~U}}1
      {ũ}{{\~u}}1    {Ū}{{\={U}}}1  {ū}{{\={u}}}1  {Ů}{{\r{U}}}1  {ů}{{\r{u}}}1
      {Ű}{{\H{U}}}1  {ű}{{\H{u}}}1  {Ų}{{\k{U}}}1  {ų}{{\k{u}}}1  {Ź}{{\'Z}}1
      {ź}{{\'z}}1    {Ż}{{\.Z}}1    {ż}{{\.z}}1    {Ž}{{\v{Z}}}1
      % ¿ and ¡ are not correctly displayed if inconsolata font is used
      % together with the lstlisting environment. Consider typing code in
      % external files and using \lstinputlisting to display them instead.      
  }

\title{Preprocesamiento de texto}
\author{Fabián Villena}
\date{Junio 2025}

\begin{document}

\maketitle

El el proveedor de salud donde usted trabaja existe un conjunto de documentos clínicos redactados por médicos y le piden que realice un análisis descriptivo de la distribución de palabras dentro de los documentos. Para llegar a este objetivo usted primero debe preprocesar el texto para que los análisis posteriores se realicen de manera más fácil.

Para desarrollar sus funciones de preprocesamiento se le entrega el siguiente texto de muestra

\begin{lstlisting}[breaklines=true, extendedchars=true,numbers=left,frame=single]
- TRASTORNO DE LA REFRACCIÓN, NO ESPECIFICADO


 paciente de 71 años, con antecedentes de hta en tto, diabetes insulinodependiente, dislipidemia, hipotiroidismo en tto, enfermedad renal cronica etapa iii,tabaquismo cronico importante, en febrero de este año lo suspendio. Refiere que tiene principios de Alzheimer y parkinson?????? NO SALE REGISTRO DE DIAGNOSTICOS. Refiere que necesita ic a oftalmologo. Tiene astigmatismo y miopia, ocupa lentes para ambos trastornos de viciorefraccion, refiere que hace 4 meses que ve borroso utilizando lentes ópticos.  Fue operada hace mas de 2 años por retinopatia diabetica en ambos ojos. 

Al ex fisico: No observo ojo rojo. pupilas isocoricas y reactivas. no observo opacidades corneales. RFM presente. agudeza visual conservada.
\end{lstlisting}

El archivo de muestra también se encuentra disponible en la siguiente dirección web:

\begin{center}
	\url{https://users.dcc.uchile.cl/~fvillena/files/text-230211.txt}
\end{center}

\section*{Preguntas}

Responda las siguientes preguntas en un \textit{Jupyter Notebook} con código desarrollado en el lenguaje de programación Python.

\begin{enumerate}
	\item Importe el archivo y asegúrese que se estén conservando todos los caracteres que no están codificados en la tabla ASCII en la cadena de caracteres importada.
	\item Desarrolle una función que preprocese el texto utilizando las técnicas que usted estime que son necesarias para que el texto se pueda analizar posteriormente de la manera más informativa posible. La función debe tener la firma \texttt{str} $\rightarrow$ \texttt{str}.
    \item Implemente una función que \textit{tokenice} el texto. La función debe tener la firma \texttt{str} $\rightarrow$ \texttt{List[str]}.
    \item Hay algunas palabras que al parecer no aportan información al significado de las oraciones, tales como las palabras \textit{de}, \textit{con} y \textit{en}. Construya una lista de palabras que usted considere que no aportan significado y filtre el texto \textit{tokenizado} eliminando estas palabras.
    \item Visualice el documento de texto utilizando una nube de palabras.
\end{enumerate}

\end{document}